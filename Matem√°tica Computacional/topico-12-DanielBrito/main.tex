\documentclass[]{article}

\usepackage[brazil]{babel}
\usepackage[utf8]{inputenc}

\usepackage{amsmath}
\usepackage{amsfonts}
\usepackage{amsthm}
\usepackage[shortlabels]{enumitem}

\usepackage{hyperref}

\usepackage{graphics}
\usepackage{tikz}
\usetikzlibrary{shapes.geometric}

\begin{document}
85) Se $z$ e $z(u)$ estão bem definidos, seus respectivos programas lineares são viáveis e limitados, logo, têm uma solução ótima. Além disso, com o vetor de parâmetros $u \geq 0$, temos que o fator de penalização, escrito em função das restrições, é não negativo. Baseado nisso, podemos concluir que o valor de $z(u)$ é, pelo menos, igual à $z$, ou seja, $z \leq z(u)$.

\vspace{0.5cm}

86) Como $Ax \leq b$ e $A^Tu \geq c$, com $x \geq 0$ e $y \geq 0$, então:

\begin{align*}
c^Tx \leq (A^Tu)^Tx = u^TAx \leq u^Tb = b^Tu
\end{align*}

\end{document}