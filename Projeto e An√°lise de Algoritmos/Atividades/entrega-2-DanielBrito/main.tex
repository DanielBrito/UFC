\documentclass[]{article}

\usepackage[a4paper, total={6in, 8in}]{geometry}

\usepackage[brazil]{babel}
\usepackage[utf8]{inputenc}

\usepackage{amsmath}
\usepackage{amsfonts}
\usepackage{amsthm}
\usepackage{listings}
\usepackage{hyperref}
\usepackage{graphics}
\usepackage{tikz}

\begin{document}

\begin{center}
  \Large\textbf{Entrega 2}\\
  \large\textit{Daniel Brito}
\end{center}

13) A ideia deste algoritmo consiste em, basicamente, utilizar a implementação do \texttt{mergeSort} para ordenar os elementos em tempo $O(n \log n)$, depois, utilizar o \texttt{removerDuplicatas}, cujo tempo é $O(n)$, e ir comparando um elemento com o seu adjacente à direita, a fim de armazenar os valores convenientes no array \texttt{resultado}.

\begin{lstlisting}
def mergeSort(array):
  if len(array) > 1:
    mid = len(array)//2
    L = array[:mid]
    R = array[mid:]
    mergeSort(L)
    mergeSort(R)
    i = j = k = 0

    while i < len(L) and j < len(R):
        if L[i] < R[j]:
            array[k] = L[i]
            i += 1
        else:
            array[k] = R[j]
            j += 1
        k += 1

    while i < len(L):
        array[k] = L[i]
        i += 1
        k += 1

    while j < len(R):
        array[k] = R[j]
        j += 1
        k += 1
      
def removerDuplicatas(array):
    n = len(array)
    resultado = []
    
    if n == 1 or n == 0:
        return array
    
    for i in range(0, n-1):
        if(array[i] != array[i+1]):
            resultado.append(array[i])

    resultado.append(array[n-1])
    return resultado
\end{lstlisting}

\newpage

14) Para resolver este problema, serão apresentados dois algoritmos, um iterarito e outro recursivo, ambos bastante intuitivos, pois utilizam a ideia de busca binária, atendendo a complexidade $O(\log n)$. A ideia consiste em realizar uma pequena modificação nos condicionais, ou seja, ao invés de procurar por um determinado valor $x$, compara-se o valor de \texttt{mid}, utilizando um lowerbound (\texttt{low}) e um upperbound (\texttt{up}) para auxiliar no processo.

\vspace{0.5cm}

I) Iterativo: Recebendo apenas o array como parâmetro.

\begin{lstlisting}
def buscaIterativa(a):
    mid = 0
    low = 0
    up = len(a)-1

    while (low <= up):
        mid = (low + up) // 2

        if mid == a[mid]:
            return True
        if mid < a[mid]:
            up = mid-1
        else:
            low = mid+1
            
    return False
\end{lstlisting}

\vspace{1cm}

I) Recursivo: Recebendo o array \texttt{a}, o lowerbound \texttt{low = 0}, e o upperbound \texttt{up = len(a)-1}.

\begin{lstlisting}
def buscaRecursiva(a, low, up):      
    if up >= low: 
        mid = (low+up) // 2
        
        if mid == a[mid]: 
            return True
        if mid > a[mid]: 
            return buscaRecursiva(a, (mid+1), up) 
        else: 
            return buscaRecursiva(a, low, (mid-1)) 
    return False
\end{lstlisting}

\newpage

15) A ideia do algoritmo \texttt{busca} consiste em encontrar o sub-array dos elementos válidos, ou seja, antes de $\infty$, por meio da utilização de índices delimitadores. Uma vez que isto ocorre, tal sub-array é passado para o método de busca binária clássico. Ambos os métodos têm como complexidade $O(\log n)$, atendendo ao requisito do problema.

\begin{lstlisting}
def busca(array, x):
    if array[0] > x:
        return False
    
    if array[0] == x:
        return True
    
    left = 1
    right = 1
    
    while(array[right] != "INF"):
        left = right
        right *= 2
        
    mid = 0
    
    while((right-left) > 1):
        mid = left + ((right-left) // 2)
        
        if(array[mid] == "INF"):
            right = mid
        else:
            left = mid
            
    return buscaBinaria(array[0:left+1], x)
\end{lstlisting}

\vspace{0.5cm}

\begin{lstlisting}
def buscaBinaria(array, x):
    mid = 0
    low = 0
    up = len(array)-1
    
    while (low <= up):
        mid = (low + up) // 2
    
        if array[mid] == x:
            return True
        if x < array[mid]:
            up = mid-1
        else:
            low = mid+1
            
    return False
\end{lstlisting}

\newpage

16) a) A ideia deste algoritmo consiste em combinar pares de arrays por meio do procedimento \texttt{merge}, o que resulta em uma complexidade final de $O(n * k * \log (n * k))$.

\begin{lstlisting}
def combinarArrays(arrays):
    k = len(arrays)
    
    if(k < 2):
        return arrays
    
    resultado = merge(arrays[0], arrays[1])
    
    for i in range(2, k):
        resultado = merge(resultado, arrays[i])
        
    return resultado
    

def merge(arr1, arr2): 
    n1 = len(arr1)
    n2 = len(arr2)
    arr3 = [None] * (n1 + n2)
    
    i = j = k = 0

    while i < n1 and j < n2: 
        if arr1[i] < arr2[j]: 
            arr3[k] = arr1[i] 
            k = k + 1
            i = i + 1
        else: 
            arr3[k] = arr2[j] 
            k = k + 1
            j = j + 1
    
    while i < n1: 
        arr3[k] = arr1[i]; 
        k = k + 1
        i = i + 1

    while j < n2: 
        arr3[k] = arr2[j]; 
        k = k + 1
        j = j + 1
        
    return arr3
\end{lstlisting}

\newpage

16) b) A versão otimizada do algoritmo anterior utiliza uma Min Heap, cuja implementação a ser apresentada foi desenvolvida em C, na disciplina de Estruturas de Dados Avançada. A ideia geral para combinar os arrays é criar uma Min Heap com cada um de seus elementos, e depois ir removendo os mesmos, convenientemente, de maneira a preencher o array \texttt{resultado}. Uma vez que temos $n * k$ elementos, e as operações de inserção é remoção da Min Heap possuem custo $O(\log n)$, no final, alcançamos uma complexidade $O(n * k * \log k)$.

\begin{lstlisting}
// Modulo da MinHeap:

heap* criar(int nmax){
	heap* h = (heap*)malloc(sizeof(heap));
	h->n=0;
	h->nmax = nmax;
	h->v = (int*)malloc(sizeof(int)*nmax);
	return h;
}

int heap_vazia (heap *h){
	return h->n == 0;
}

void sobe (heap *h, int i){
	int pai;
	while(i>0){
		pai = pai(i);
		if(h->v[pai] <= h->v[i]) break;
		troca(h, pai, i);
		i = pai;
	}	
}

void troca (heap *h, int i, int j){
	int temp = h->v[i];
	h->v[i] = h->v[j];
	h->v[j] = temp;
}

void desce (heap *h, int i){
	int filhoEsquerda = esq(i);
	int filhoDireita = dir(i);
	while(filhoEsquerda < h->n){
		filhoDireita = dir(i);
		if((filhoDireita > h->n) && 
		   (h->v[filhoDireita] < h->v[filhoEsquerda])){
			filhoEsquerda = filhoDireita;
		}
		if(h->v[filhoEsquerda] > h->v[i]){
			break;
		}
		troca(h, i, filhoEsquerda);
		i = filhoEsquerda;
		filhoEsquerda = esq(i);
	}
}

void heap_insere(heap *h, int valor){
	h->v[h->n++] = valor;
	sobe(h, h->n-1);		
}

int heap_retira (heap *h){
	int raiz = h->v[0];
	h->v[0] = h->v[--h->n];
	desce(h, 0);
	return raiz;	
}

// Programa principal:

int main(void) {
	int i, j, r, maximo_elementos, n = 3;
	int arrays[n][n] = {{1, 3, 5}, {2, 4, 8}, {7, 10, 21}};
	maximo_elementos = n*n;
	r = 0;
	int resultado[maximo_elementos];
	heap *h = criar(maximo_elementos);

	for(i=0; i<n; i++){
		for(j=0; j<n; j++){
			heap_insere(h, arrays[i][j]);
		}
	}

	while(!heap_vazia(h)){
		resultado[r++] = heap_retira(h);
	}

    return 0;
}
\end{lstlisting}


\newpage

17) Neste algoritmo é realizada a verificação de alguns casos base, então, se nenhum deles é atingido, são feitas chamadas recursivas passando-se os segmentos dos arrays para a procura o k-ésimo elemento.

\begin{lstlisting}
def kesimoMenorElemento(arr1, arr2, x1, x2, k):
    if arr1[0] == x1:
        return arr2[k]
    
    if arr2[0] == x2:
        return arr1[k]
    
    mid1 = (x1 - arr1[0]) // 2
    mid2 = (x2 - arr2[0]) // 2
    
    if (mid1 + mid2) < k:
        if arr1[mid1] > arr2[mid2]:
            return kesimoMenorElemento(arr1, arr2[:mid2+1], x1, x2, k-mid2-1)
        else:
            return kesimoMenorElemento(arr1[:mid1+1], arr2, x1, x2, k-mid1-1)
    else:
        if arr1[mid1] > arr2[mid2]:
            return kesimoMenorElemento(arr1, arr2, arr1[mid1], x2, k)
        else:
            return kesimoMenorElemento(arr1, arr2, x1, arr2[mid2], k)
\end{lstlisting}

\newpage

18) a) Neste algoritmo, o array \texttt{a} é dividido em dois segmentos, então, uma chama recursiva é realizada em ambas a partes para verificar a existência do elemento majoritário.

\begin{lstlisting}
def elementoMajoritario(a):
    n = len(a)
    
    if n == 1:
        return a[0]
    if n == 0:
        return None
    
    k = n // 2
    
    elem1 = elementoMajoritario(a[:k])
    elem2 = elementoMajoritario(a[k+1:])
    
    if elem1 == elem2:
        return elem1

    count1 = a.count(elem1)
    count2 = a.count(elem2)

    if count1 > k:
        return elem1
    elif count2 > k:
        return elem2
    else:
        return None
\end{lstlisting}

\newpage

18) b) Este algoritmo segue a ideia do enunciado, ou seja, dividir o array em $\frac{n}{2}$ pares, depois, realizar o procedimento de "armazenamento" de um dos valores, se ambos forem iguais, ou "descarte", se forem diferentes. Por fim, é verificada a existência do elemento majoritário.

\begin{lstlisting}
def elementoMajoritarioB(a, e=None):
    n = len(a)
     
    if n == 0:
        return e
     
    pares = []
     
    if n % 2 == 1:
        e = a[-1]
         
    for i in range(0, n-1, 2):
        if a[i] == a[i+1]:
            pares.append(a[i])
            
    majoritario = elementoMajoritarioB(pares, e)
     
    if majoritario is None:
        return None
     
    nMajoritario = a.count(majoritario)
     
    if 2 * nMajoritario > n or (2*nMajoritario == n and majoritario == e):
        return majoritario
     
    return None
\end{lstlisting}

\vspace{1cm}

18) c) Dada a relação de recorrência T(n) = T(n/2) + O(n), com base no Teorema Mestre, temos que T(n) = O(n).

\end{document}