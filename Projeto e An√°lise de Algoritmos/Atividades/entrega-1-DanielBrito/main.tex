\documentclass[]{article}

\usepackage[a4paper, total={6in, 8in}]{geometry}

\usepackage[brazil]{babel}
\usepackage[utf8]{inputenc}

\usepackage{amsmath}
\usepackage{amsfonts}
\usepackage{amsthm}
\usepackage{listings}
\usepackage{hyperref}
\usepackage{graphics}
\usepackage{tikz}

\begin{document}

\begin{center}
  \Large\textbf{Entrega 1}\\
  \large\textit{Daniel Brito}
\end{center}

\vspace{0.5cm}

1) Dado um grafo direcionado $G=(V, A)$, sendo representado por uma matriz de adjacência, o algoritmo abaixo decide se $G$ possui uma celebridade. A ideia consiste em encontrar um candidato por meio da função \textit{encontrarCandidato}, e depois verificar se ele é realmente uma celebridade ou não, por meio da função \textit{verificarCelebridade}, checando se o candidato conhece ninguém e todos o conhecem. Se o candidato for uma celebridade a função retorna \textit{True}, caso contrário, retorna \textit{False}.

\begin{lstlisting}
    def encontrarCandidato(G, n):
	if(n==1): 
	  return n-1

	candidato = encontrarCandidato(G, n-1) 
	
	if(candidato==-1): 
	  return n-1
    
	elif(G[candidato][n-1] and not(G[n-1][candidato])): 
	  return n-1
    
	elif(G[n-1][candidato] and not(G[candidato][n-1])): 
	  return candidato 

	return -1

    def verificarCelebridade(G, n): 
        
      candidato = encontrarCandidato(G, n)
        
      if(candidato==-1):
        return False
    	
      u = v = 0
    
      for i in range(n): 
        if (i!=candidato): 
          u += G[candidato][i] 
          v += G[i][candidato]
    
      if(u==0 and v==n-1): 
        return True
      else:
        return False
\end{lstlisting}

\newpage

2) Para mostrar que $(n+a)^b = \Theta(n^b)$, devemos encontrar constantes $c_1, c_2, n_0 > 0$, tal que $0 \leq c_1 \cdot n^b \leq (n+a)^b \leq c_2 \cdot n^b$ para todo $n \geq n_0$. Assim, temos:

\begin{align*}
    n+a & \leq n+|a| \\
        & \leq 2n, |a| \leq n
\end{align*}

e

\begin{align*}
    n+a & \geq n-|a| \\
        & \geq \frac{1}{2}n, |a| \leq \frac{1}{2}n
\end{align*}

Portanto, quando $n \geq 2 \cdot |a|$, temos:

\begin{align*}
    0 \leq \frac{1}{2}n \leq n+a \leq 2n.
\end{align*}

Como $b>0$, a inequação ainda é válida, ou seja:

\begin{align*}
    0 \leq (\frac{1}{2}n)^b \leq (n+a)^b \leq (2n)^b, \\
    0 \leq (\frac{1}{2})^b \cdot n^b \leq (n+a)^b \leq 2^b \cdot n^b.
\end{align*}

Por fim, temos que $c_1 = (\frac{1}{2})^b$, $c_2 = 2^b$ e $n_0 = 2 \cdot |a|$, satisfazendo a definição.

\vspace{1cm}

3) Sim. Para mostrar que $2^{n+1} \in O(2^n)$, devemos encontrar constantes $c, n_0 > 0$, tal que $0 \leq 2^{n+1} \leq c \cdot 2^n$, para todo $n \geq n_0$. Como $2^{n+1} = 2 \cdot 2^n$ para todo $n \geq 0$, temos que a definição é satisfeita com $c=2$ e $n_0=1$.

\vspace{0.5cm}

Não. Note que $2^{2n} = 2^n \cdot 2^n$. Desta maneira, para $2^{2n} \in O(2^n)$, precisamos de uma constante $c$, tal que $0 \leq 2^n \cdot 2^n \leq c \cdot 2^n$. Fica evidente que precisamos de um $c \geq 2^n$. Entretanto, isso não é possível para um valor de $n$ arbitrariamente alto. Ou seja, não importa o valor escolhido para $c$, para algum valor suficientemente alto de $n$, c não será suficiente. Portanto, concluímos que $2^{2n} \notin O(2^n)$.

\vspace{1cm}

4) Utilizando as definições de limite, $o(n)$ e $\omega(n)$, temos que:

\begin{align*}
    o(g(n)) = \lim_{n \to \infty} \frac{f(n)}{g(n)} = 0   
\end{align*}

e

\begin{align*}
    \omega(g(n)) = \lim_{n \to \infty} \frac{f(n)}{g(n)} = \infty   
\end{align*}

Entretanto, ambos os casos não se mantêm verdadeiros quando $n$ se aproxima de $\infty$. Desta maneira, a interseção é vazia.

\vspace{1cm}

5) Sejam os itens:

a) $n! \in \omega(2^n)$:

\begin{align*}
    n! & = n \cdot (n-1) \cdot ... \cdot 2 \cdot 1 \\
       & > 2 \cdot 2 \cdot ... \cdot 2, n \geq 4 \\
       & = 2^n
\end{align*}

Desta forma, existe uma constante positiva $c=1$ e $n_0=4$, na qual $n! > c \cdot 2^n$, para todo $n \geq n_0$.

\vspace{0.5cm}

b) $n! \in o(n^n)$:

% \vspace{0.5cm}

% Para provar que $n! = o(n^n)$, temos:

% \begin{align*}
%     n! & = n \cdot (n-1) \cdot ... \cdot 2 \cdot 1 \\
%       & < n \cdot n \cdot ... \cdot n, n \geq 2 \\
%       & = n^n
% \end{align*}

% Desta forma, existe uma constante positiva $c=1$ e $n_0=2$, na qual $n! < c \cdot n^n$, para todo $n \geq n_0$.

% \vspace{1cm}

Com base na definição, temos $0 \leq f(n) < c \cdot g(n)$, para todo $c>0$ e $n \geq n_0$. Seja $c=1$ e $n_0=1$.

\begin{align*}
    \text{Caso Base}: \quad & n = 1 \\
                      \quad & n! \leq n^n \\
                      \quad & 1! \leq 1^1 \\
                      \quad & 1 \leq 1 \\
    \text{Hipótese de indução:} \quad & \text{Assuma que } k! \leq k^k, k>1 \\
    \text{Passo indutivo:} \quad & \text{Queremos provar que } (k+1)! \leq (k+1)^{(k+1)}. \\
                              k! & \leq k^k \\
                       (k+1)(k!) & \leq (k+1) \cdot (k^k) < \\
                       (k+1)! \leq 
                                   & \leq c(k+1) + 1 \\
                                   & = c(k+1) + O(1) \\
                                   & = c(k+1) \\
                                   & = O(n)
\end{align*}

Desta forma, existe uma constante positiva $c=1$ e $n_0=2$, na qual $n! < c \cdot n^n$, para todo $n \geq n_0$.

\vspace{1cm}

6) Sejam os itens:

\vspace{0.5cm}

a) Se $ k \geq d$, então $p(n) \in O(n^k)$:

Como $k \geq d$, $n^k$ cresce mais rápido ou no mesmo ritmo que $n^d$ para um $n$ suficientemente grande. Desta maneira, $p(n) = O(n^k)$. 

Para $c$, temos $0 \leq p(n) \leq 1.5 \cdot a_d \cdot n^d \leq 1.5 \cdot a_d \cdot n^k$.

Assim, se tomarmos $c_1 = 1.5 \cdot a_d$, temos $0 \leq p(n) \leq c_1 \cdot n^k$, ou seja, $p(n) = O(n^k)$.

\vspace{0.5cm}

b) Se $ k \leq d$, então $p(n) \in \Omega(n^k)$:

Como $k \leq d$, $n^k$ cresce mais devagar ou no mesmo ritmo que $n^d$ para um $n$ suficientemente grande. Desta maneira, $p(n) = \Omega(n^k)$. 

Para $c$, temos $0 \leq p(n) \leq 0.5 \cdot a_d \cdot n^d \leq 0.5 \cdot a_d \cdot n^k$.

Assim, se tomarmos $c_1 = 0.5 \cdot a_d$, temos $0 \leq p(n) \leq c_1 \cdot n^k$, ou seja, $p(n) = \Omega(n^k)$.

\vspace{0.5cm}

c) TO-DO

d) TO-DO

e) TO-DO

\newpage

7) Ordenação crescente, sendo que as funções na mesma linha possuem classe de equivalência iguais:

\begin{align*}
    1, n^{\frac{1}{log(n)}} \\
    2^{log(n)} \\
    ln(ln(n)) \\
    \sqrt{log(n)} \\
    ln(n) \\
    log^2(n) \\
    2^{\sqrt{2 \cdot log(n)}} \\
    (\sqrt{2})^{log(n)} \\
    n \\
    n \cdot log(n) \\
    4^{log(n)}, n^2 \\
    n^3 \\
    n^{log(log(n))}, log(n)^{log(n)} \\
    log(n!) \\
    (\frac{3}{2})^n \\
    2^n \\
    e^n \\
    n \cdot 2^n \\
    n! \\
    (n+1)! \\
    2^{2^n} \\
    2^{2^{n+1}}
\end{align*}

\newpage

8) Sejam os itens:

a) Falso. Contra-exemplo: Seja $f(n)=n$ e $g(n)=n^2$. Assim, temos que $n = O(n^2)$, mas $n^2 \neq O(n)$.

\vspace{0.5cm}

b) Falso. Contra-exemplo: Seja $f(n)=n$ e $g(n)=n^2$, mas $n^2 + n \neq \Theta(n)$.

\vspace{0.5cm}

c) Verdadeiro. Prova: $f(n) = O(g(n))$, ou seja, $0 \leq f(n) \leq c \cdot g(n)$, para todo $n \geq n_0$, com $c, n_0 > 0$. Desta forma, $0 \leq log(f(n)) \leq log(c) + log(g(n)) \leq k \cdot log(g(n))$. Portanto, $log(f(n)) = O(log(g(n))$.

\vspace{0.5cm}

d) Falso. Contra-exemplo: Seja $f(n)=2n$ e $g(n)=n$. Desta forma, $f(n) = O(g(n))$, mas $2^{2n} = 4^n \neq O(2^n)$.

\vspace{1cm}

9) Sejam os itens:

a) Tomando como estimativa $T(n) \leq cn = O(n)$.

\begin{align*}
    \text{Caso Base}: \quad & n = 1 \\
                      \quad & T(1) = 1 \\
    \text{Hipótese de indução:} \quad & T(k) \leq ck, \text{para } c>0, k>1 \\
    \text{Passo indutivo:} \quad & \text{Queremos provar que } T(k+1) \leq c(k+1) = O(n). \\
                            T(k+1) & = T(k+1-1)+1 \\
                                   & = T(k)+1 \\
                                   & \leq c(k+1) + 1 \\
                                   & = c(k+1) + O(1) \\
                                   & = c(k+1) \\
                                   & = O(n)
\end{align*}

b) Tomando como estimativa $T(n) \leq cn^2 = O(n^2)$.

\begin{align*}
    \text{Caso Base}: \quad & n = 1 \\
                      \quad & T(1) = 1 \\
    \text{Hipótese de indução:} \quad & T(k) \leq ck^2, \text{para } c>0, k>1 \\
    \text{Passo indutivo:} \quad & \text{Queremos provar que } T(k+1) \leq c(k+1)^2 = O(n^2). \\
                            T(k+1) & = T(k+1-1)+k \\
                                   & = T(k)+k \\
                                   & \leq c(k+1)^2+k \\
                                   & = c(k+1)^2 + O(n) \\
                                   & = c(k+1)^2 \\
                                   & = O(n^2)
\end{align*}

\newpage

c) Para esta recorrência, temos $a=1$, $b=2$, $f(n)=1$, e $n^{\log_b^a} = n^{\log_2^1} = n^0$. Então, podemos aplicar o caso 2 do Teorema Mestre. Assim:

\begin{align*}
    T(n) = O(n^{\log_b^a} \log n) = O(n^0 \cdot \log n) = O(\log n).
\end{align*}

\vspace{0.5cm}

d) Para esta recorrência, temos $a=3$, $b=2$, $f(n)=n$, e $n^{\log_b^a} = n^{\log_2^3} \simeq n^{1.585}$ (TO-DO)

\vspace{0.5cm}

e) Para esta recorrência, temos $a=4$, $b=3$, $f(n)=n$, e $n^{\log_b^a} = n^{\log_3^4} \simeq n^{0.792}$ (TO-DO)

\vspace{0.5cm}

f) Para esta recorrência, temos $a=4$, $b=2$, $f(n)=n^2$, e $n^{\log_b^a} = n^{\log_2^4} = n^2$. Então, podemos aplicar o caso 2 do Teorema Mestre. Assim:

\begin{align*}
    T(n) = O(n^{\log_b^a} \log n) = O(n^2 \cdot \log n).
\end{align*}

\vspace{0.5cm}

g) TO-DO

\vspace{1cm}

10) Sejam os itens:

a) Para esta recorrência, temos $a=2$, $b=4$, $f(n)=1$, e $n^{\log_b^a} = n^{\log_4^2}$. Como $f(n)=1 = O(n^{\log_4^{2-\epsilon}})$, onde $\epsilon = 0.2$, podemos aplicar o caso 1 do Teorema Mestre, e concluir que:

\begin{align*}
    T(n) = \Theta(n^{\log_4^2} \log n) = \Theta(n^{\frac{1}{2}}) = \Theta(\sqrt{n}).
\end{align*}

\vspace{0.5cm}

b) Para esta recorrência, temos $a=2$, $b=4$, $f(n)=\sqrt{n}$, e $n^{\log_b^a} = n^{\frac{1}{2}} = \sqrt{n}$. Como $f(n)=\Theta(\sqrt{n})$, podemos aplicar o caso 2 do Teorema Mestre, e concluir que:

\begin{align*}
    T(n) = \Theta(\sqrt{n} \cdot \log n).
\end{align*}

\vspace{0.5cm}

c) Para esta recorrência, temos $a=2$, $b=4$, $f(n)=n$, e $n^{\log_b^a} = n^{\log_4^2}$. Como $f(n)=\Omega(n^{\log_4^{2+\epsilon}})$, onde $\epsilon = 0.2$, podemos aplicar o caso 3 do Teorema Mestre. Assim, temos que provar que $af(\frac{n}{b}) \leq cf(n)$, para alguma constante $c<1$, e todos os valores suficientemente altos de $n$. Logo, como $af(\frac{n}{b}) = 2(\frac{n}{4}) = \frac{n}{2} \leq cf(n)$, para $c=0.7$ e $n \geq 2$, podemos concluir que:

\begin{align*}
    T(n) = \Theta(f(n)) = \Theta(n).
\end{align*}

\vspace{0.5cm}

d) Para esta recorrência, temos $a=2$, $b=4$, $f(n)=n^2$, e $n^{\log_b^a} = n^{\log_4^2}$. Como $f(n)=\Omega(n^{\log_4^{2+\epsilon}})$, onde $\epsilon = 1$, podemos aplicar o caso 3 do Teorema Mestre. Assim, temos que provar que $af(\frac{n}{b}) \leq cf(n)$, para alguma constante $c<1$, e todos os valores suficientemente altos de $n$. Logo, como $af(\frac{n}{b}) = 2(\frac{n}{4})^2 = (\frac{1}{8})n^2 \leq cf(n)$, para $c=0.5$ e $n \geq 4$, podemos concluir que:

\begin{align*}
    T(n) = \Theta(n^2).
\end{align*}

\vspace{0.5cm}

e) Para esta recorrência, temos $a=1$, $b=2$, $f(n)=1$, e $n^{\log_b^a} = n^{\log_2^1} = n^0$. Então, podemos aplicar o caso 2 do Teorema Mestre. Assim:

\begin{align*}
    T(n) = \Theta(n^{\log_b^a} \log n) = \Theta(n^0 \cdot \log n) = \Theta(\log n).
\end{align*}

\vspace{1cm}

11) Para esta recorrência, temos $a=4$, $b=2$, $f(n)=n^2 \cdot \log n$, e $n^{\log_b^a} = n^{\log_2^4} = n^2$. Assintoticamente falando, $f(n)=n^2 \cdot \log n > n^2$, mas não polinomialmente. Portanto, não podemos aplicar o Teorema Mestre neste caso. (TO-DO)

\vspace{1cm}

12) Para escolher qual dos algoritmos utilizar, precisamos fazer a seguinte análise:

\vspace{0.5cm}

1) $5T(\frac{n}{2}) + n$

Para esta recorrência, temos $a=5$, $b=2$, $f(n)=n$, e $n^{\log_b^a} = n^{\log_2^5} = O(n^{2.322})$.

\vspace{0.5cm}

2) $2T(n-1) + 1$

Tomando como estimativa $T(n) \leq 2^n = O(2^n)$.

\begin{align*}
    \text{Caso Base}: \quad & n = 1 \\
                      \quad & T(1) \leq 2 \\
    \text{Hipótese de indução:} \quad & T(k) \leq 2^k, \text{para } k>1 \\
    \text{Passo indutivo:} \quad & \text{Queremos provar que } T(k+1) \leq 2^{(k+1)} = O(2^{(k+1)}). \\
                            T(k+1) & = 2T(k+1-1)+1 \\
                                   & = 2T(k)+1 \\
                                   & \leq 2 \cdot 2^k + 1 \\
                                   & = 2^{(k+1)} + O(1) \\
                                   & = 2^{(k+1)} \\
                                   & = O(2^{(k+1)})
\end{align*}

\vspace{0.5cm}

3) $9T(\frac{n}{3}) + n^2$

Para esta recorrência, temos $a=9$, $b=3$, $f(n)=n^2$, e $n^{\log_b^a} = n^{\log_3^9} = O(n^2 \log n)$.

\vspace{0.5cm}

Portanto, com base nesta análise, o algoritmo (1) deveria ser escolhido para resolver o problema.

\end{document}